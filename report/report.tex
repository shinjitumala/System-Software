% template2.tex
%
% このソースファイルをコピーして自分のレポート内容にするとよい.
%     $ cp template2.tex report2.tex

% LaTeX一式を自分のパソコンにインストールするのであればTeXLiveをおすすめする.
%     https://texwiki.texjp.org/?TeX%20Live
% Macの場合はTeXLiveをMac向けにカスタマイズしたMacTeXがよい.
%     https://texwiki.texjp.org/?MacTeX

% コマンドラインでのLaTeXの使い方
% 参考文献等の参照が正しく行われるよう,latexを2回実行する.
% report3.dviができたら,dvipdfmxでPDFに変換する.
%     $ platex report3
%     $ platex report3
%     $ dvipdfmx report3
% これ以外にも日本語に対応したLaTeX処理系(lualatex等)を使ってもよい.

% 文書スタイルの指定
% 比較的ページ数の少ない講義レポートの場合,jreport ではなく jarticle がよい.
% jsarticle が用意されていればそれを使ってもよい.
% 用紙サイズはA4とする.オプションにa4ではなくa4jを指定した方が余白が適切になる.
\documentclass[10pt,a4j]{jarticle}

% 使用するパッケージの指定

% 各種数学記号や数式を書くためのパッケージ.問答無用で入れておくとよい.
\usepackage{amsmath,amssymb}
% 旧版のLaTeXで用意されていた各種記号のためのパッケージ.
% 時相論理の□(\Box)や◇(\Diamond)をバランスよく出すために必要.
\usepackage{latexsym}

% screen, itemboxを使うために必要
\usepackage{ascmac}

% urlを参照するのに便利
\usepackage{url}

% プログラムの予約語をボールド(太字)にして目立たせると見やすい.残念ながら
% LaTeXデフォルトの等幅フォント(\textttで出てくるフォント)はボールドにならないので,
% 以下のいずれかのパッケージを使うとよい.
\usepackage{courier}  % 等幅フォントとしてCourierフォントを使う
% \usepackage{inconsolata} % 等幅フォントとしてInconsolataを使う
% \usepackage[cmbtt]{bold-extra}  % CMフォントのタイプライター体のボールドを使う

% 文書にプログラムリストを入れたいときに便利
\usepackage{listings}
% listingsパッケージのデフォルトを設定する
\lstset{%
  language={C},             % 言語をCにする
  basicstyle={\ttfamily},   % プログラムの基本スタイルは等幅(tt)とする
  keywordstyle={\bfseries}, % 予約語は太字(bf)にする
  numbers=left,             % 左側に行番号を入れる
  frame=single,             % 枠で囲む
  lineskip=-0.75ex,         % プログラムリストの改行幅を少し小さくする
  captionpos=b              % キャプションの位置をプログラムの下にする
}
\def\lstlistingname{プログラム}

% includegraphics で文書に図を入れたいときに必要
% dvipdfmx
\usepackage[dvipdfmx]{graphicx}


% 他に必要なパッケージがあったらここに追加する.

% 文書のタイトル
% レポートの場合は科目名と課題名
\title{システムソフトウェア・小課題2}

% 学籍番号と氏名(自分の学籍番号と氏名)
\author{16-9999-9・情工 太郎}

% 提出日(締切日ではなく実際の提出日)
\date{2018年11月X日}

% 本文の開始

\begin{document}

\maketitle

\section{xv6のスピンロック実装における\texttt{xchg}の役割について}

\subsection{\texttt{xchg}を使用せず,CPUコア数を1として実行した結果}

% spinlock.c で定義されている関数 acquire 中の32-33行目を以下のように変更してxv6をビルドする.
%   while (lk->locked != 0)
%     ;
%   lk->locked = 1;

% CPUコア数を1にして実行した結果を載せる.実行結果のスクリーンショットは
%     make CPUS=1 qemu-nox
% で実行した結果をターミナルからコピー&ペーストし,verbatim環境で囲めばよい.
% さらにscreen環境で囲むと実行結果らしくなる.

% 以下は変更を加えていないxv6の起動例
\begin{screen}
\begin{verbatim}
cs-mac80:System-Software hoshino.s.af$ make qemu-nox CPUS=1
Booting from Hard Disk..xv6...
cpu0: starting 0
sb: size 1000 nblocks 941 ninodes 200 nlog 30 logstart 2 inodestart 32 bmap start 58
init: starting sh
$
\end{verbatim}
\end{screen}

実行結果は、上のようになり、コア数1では正常に起動した。

次に、usertestを実行してみた結果が以下の通りである。

\begin{screen}
    \begin{verbatim}
      $ usertests
      usertests starting
      ...(中略)...
      ALL TESTS PASSED
      $
    \end{verbatim}
\end{screen}
% xv6が正常に起動しているようであれば,適当なプログラムを実行した結果を載せる.

\subsection{CPUコア数を2以上にして実行した結果}

% 以下のようにCPUコア数を2以上にして実行した結果を載せる.
%   make CPUS=2 qemu-nox
%   make CPUS=3 qemu-nox
%          :
%   make CPUS=8 qemu-nox

% 実行結果が同様であればスクリーンショット等は適宜省略してよい.
% xv6が正常に起動しているようであれば,適当なプログラムを実行した結果を載せる.
\begin{screen}
  \begin{verbatim}
    cs-mac80:System-Software hoshino.s.af$ make qemu-nox CPUS=2
    Booting from Hard Disk..xv6...
    cpu1: starting 1
    QEMU: Terminated
    cs-mac80:System-Software hoshino.s.af$ make qemu-nox CPUS=3
    Booting from Hard Disk..xv6...
    cpu1: starting 1
    cpu2: starting 2
    QEMU: Terminated
    cs-mac80:System-Software hoshino.s.af$ make qemu-nox CPUS=4
    Booting from Hard Disk..xv6...
    cpu1: starting 1
    cpu2: starting 2
    cpu3: starting 3
    QEMU: Terminated
    cs-mac80:System-Software hoshino.s.af$ make qemu-nox CPUS=5
    ...(中略)...
    cs-mac80:System-Software hoshino.s.af$ make qemu-nox CPUS=8
    Booting from Hard Disk..xv6...
    cpu1: starting 1
    cpu2: starting 2
    cpu3: starting 3
    cpu4: starting 4
    cpu5: starting 5
    cpu6: starting 6
    cpu7: starting 7
    cpu0: starting 0
    QEMU: Terminated
    cs-mac80:System-Software hoshino.s.af$
  \end{verbatim}
\end{screen}

結果として、CPUコア数が2以上になると、30秒ほど待ったとしても、
最後の1個のコアが起動をせずに止まったままになってしまう。
何度か起動しようとしていると、CPUコア数が5以上の時には時々、
cpu0が起動することがあるが、それが起動したとしても、固まったまま動かなかった。

\subsection{上記の結果になった理由}

% 上記のような結果を得た理由について調査し,その結果を書く.
% 例えば正常に起動しなかった場合,ソースコード中のどの部分でどのような理由で
% エラーになったか等を調べる.その際にデバッガによる実行結果を掲載してもよい.

上記の結果をまとめると、CPUコア数が1より大きい時に問題が生じるということだった。
その原因としては、スピンロックの状態を確認する lk->locked != 0 という行と
 lk->locked = 1; という行が別々のCPUで別々に実行される可能性が存在することである。
複数のCPUでlk->lockedが0であることを確認した後に、同時にlk->lockedを1してしまうと、
ロックを複数のCPUで確保することになり、デッドロックが起こる。

逆に、xchgを呼び出している時は、アセンブリ命令でxchgを用いている。
x86におけるxchg命令を用いると、ハードウェアにロックをさせることができ\cite{xchg}、
そのことを用いて、ロックの確保を確実に一つのCPUしかできないようにしている。

従って、ロックの確保をハードウェアレベルで排他していないと正しく実装できないということである。
%
% % 説明にソースコードやその一部が必要な場合は,以下のようにする.
% \begin{lstlisting}
% static inline uint
% xchg(volatile uint *addr, uint newval)
% {
%   uint result;
%
%   // The + in "+m" denotes a read-modify-write operand.
%   asm volatile("lock; xchgl %0, %1" :
%                "+m" (*addr), "=a" (result) :
%                "1" (newval) :
%                "cc");
%   return result;
% }
% \end{lstlisting}
%
% % 説明に図が必要な場合は以下のようにする.
% % 図は適当な作図ツールで作成し,PDF, PNG, SVG等として出力したものを用いるとよい.
\begin{figure}
\centering % 図をセンタリングする
% proc.pdf というファイルを図として挿入している.
% (この例では図の幅を文書の幅の半分にしている)
\includegraphics[width=0.5\textwidth]{proc.pdf}
% figure環境を使う場合は図にキャプションをつけること.
% また,\labelコマンドでキャプションに適当なラベル(この例ではfig:xv6proc)を
% つけて,本文中で参照するのに用いる(例えば \ref{fig:xv6proc} が図の番号になる).
\caption{xv6のプロセスの状態}\label{fig:xv6proc}
\end{figure}


\section{xv6のスピンロック実装における割込みの扱いについて}

\subsection{\texttt{pushcli}/\texttt{popcli}を削除した場合の実行結果}

% acquireとreleaseにおけるpushcliおよびpopcliの呼び出しを削除(コメントアウト)して
% ビルドし,CPUコア数をいろいろ変えて実行した結果を載せる

\begin{screen}
  \begin{verbatim}
    cs-mac80:System-Software hoshino.s.af$ make qemu-nox CPUS=1
    Booting from Hard Disk..xv6...
    cpu0: starting 0
    lapicid 0: panic: mycpu called with interrupts enabled

     8010328f 80103d8d 80103585 80102a74 80102bb1 0 0 0 0 0QEMU: Terminated
    cs-mac80:System-Software hoshino.s.af$ make qemu-nox CPUS=2
    Booting from Hard Disk..xv6...
    cpu1: starting 1
    lapicid 1: panic: mycpu called with interrupts enabled

      8010328f 80103d8d 80103585 80102a74 80102a8e 705a 0 0 0 0
    ...(中略)...
    cs-mac80:System-Software hoshino.s.af$ make qemu-nox CPUS=8
    Booting from Hard Disk..xv6...
    cpu1: starting 1
    lapicid 1: panic: mycpu called with interrupts enabled

     8010328f 80103d8d 80103585 80102a74 80102a8e 705a 0 0 0 0QEMU: Terminated
    cs-mac80:System-Software hoshino.s.af$
  \end{verbatim}
\end{screen}

実行結果は、上のようにpanicが呼び出され、CPUコア数に依らず、起動しない状態になってしまった。

\subsection{\texttt{pushcli}/\texttt{popcli}をそれぞれ\texttt{cli}/\texttt{sti}に置き換えた場合の実行結果}

% acquireとreleaseにおけるpushcliおよびpopcliの呼び出しをcliおよびsitに置き換えて
% ビルドし,CPUコア数をいろいろ変えて実行した結果を載せる

\begin{screen}
  \begin{verbatim}
    cs-mac80:System-Software hoshino.s.af$ make qemu-nox CPUS=1
    Booting from Hard Disk..xv6...

    (チラつき)
    press Ctrl-B to configure iPXE (PCI 00:03.0)...
    cs-mac80:System-Software hoshino.s.af$ make qemu-nox CPUS=2
    QEMU: Terminated
    Booting from Hard Disk..xv6...
    cpu1: starting 1
    lapicid 1: panic: mycpu called with interrupts enabled

     8010328f 80102a61 80102a8e 705a 0 0 0 0 0 0QEMU: Terminated
    cs-mac80:System-Software hoshino.s.af$ make qemu-nox CPUS=3
    Booting from Hard Disk..xv6...
    cpu1: starting 1
    lapicid 1: panic: mycpu called with interrupts enabled

     8010328f 80102a61 80102a8e 705a 0 0 0 0 0 0QEMU: Terminated
    ...(中略)...
    cs-mac80:System-Software hoshino.s.af$ make qemu-nox CPUS=8
    Booting from Hard Disk..xv6...
    cpu1: starting 1
    lapicid 1: panic: mycpu called with interrupts enabled

     8010328f 80102a61 80102a8e 705a 0 0 0 0 0 0QEMU: Terminated
    cs-mac80:System-Software hoshino.s.af$
  \end{verbatim}
\end{screen}

CPUコア数が1の時は、"Booting from Hard Disk..xv6..."まで表示され、
"press Ctrl-B to configure iPXE (PCI 00:30.0)..."だと思われる文字列が
チラついて表示された。

一方で、CPUコア数が2以上の時は、panicが呼び出されて、起動しなかった。

\subsection{上記の結果になった理由}

% 上記のような結果を得た理由について調査し,その結果を書く.
% 例えば正常に起動しなかった場合,ソースコード中のどの部分でどのような理由で
% エラーになったか等を調べる.その際にデバッガによる実行結果を掲載してもよい.

まず、pushcliとpopcliをコメントアウトした時の実行結果について説明する。

pushcliとpopcliは、割り込み禁止を管理している。
もし、あるhogeというプロセスがlock0というロックを確保したあとに、
fogeというもう一つのプロセスが割り込みをし、lock0のロックを確保しようとすると、
hogeは割り込まれた状態のままで実行されず、lock0を解放できないので、
fogeもlock0の解放を待ち、デッドロックに繋がる状況になってしまう。
よって、pushcliとpopcliを用いてロックを確保する時に割り込みを禁止する必要があるが、
それをコメントアウトするとデッドロックが生じるはずである。今回、出力の結果としては、
panicが呼び出されるということだったが、それはxv6側で不本意な割り込みに対してエラー処理を
しているからである。


次に、pushcliとpopcliをそれぞれcliとstiにした時の実行結果について説明する。

cliとstiは、それぞれ割り込みを禁止、許可するハードウェア命令である。
pushcliとpopcliを用いているときは、pushcliを呼び出した回数だけpopcliを
呼び出さないと割り込みの禁止が解放されないという実装になっている\cite{doc}。
従って、cliとstiを代わりに使用した場合、一つのCPUで複数のプロセスがロックを
確保している時に、ロックの確保で正しく割り込みが禁止されるが、プロセスのうちの
一つでもロックを解放すると、割り込み許可状態になり、ロックを確保したプロセスが
割り込まれてデッドロックが起きる危険性が生じる。それにより、pushcliとpopcliを
コメントアウトした時と同じような現象が起こる。

\section{感想・意見}

% もし感想や意見があればここに書く.考察と感想は異なるので注意.
% なければ無理に書く必要はないが,もし講義をより良くするための建設的な意見が
% あったら遠慮なく書いていただきたい.
% レポートの採点とは無関係であることを約束する.

% 参考文献の例
% 参考にした図書・雑誌記事・論文やwebサイト等を明記する.
% 各文献情報は,他人が参照できるよう,以下の情報を明記すること.
% 図書: 著者, タイトル, 出版社, 出版年
% 雑誌記事: 著者, タイトル, 誌名, 号, ページ(開始ページと終了ページ), 出版社, 出版年
% webサイト: 著者, タイトル, URL, アクセスした年月日, 書かれた年月日等

\begin{thebibliography}{10}

% 各文献の記述は \bibitem{キーワード} で始める.
% キーワードは,本文中で文献を参照するときに用いる.
% 具体的には,\cite{キーワード} のようにすると,この部分が文献番号になる.

\bibitem{xv6}
Russ Cox, Frans Kaashoek \& Robert Morris,
``xv6, a simple Unix-like teaching operating system'', Sep., 2018,
\url{https://pdos.csail.mit.edu/6.828/2018/xv6/book-rev11.pdf}.

\bibitem{xchg}
Intel,
``Intel® 64 and IA-32 Architectures Software Developer's Manual, page 2396 and 2397'', Nov., 2018,
\url{https://software.intel.com/sites/default/files/managed/39/c5/325462-sdm-vol-1-2abcd-3abcd.pdf}

\bibitem{doc}
Russ Cox, Frans Kaashoek \& Robert Morris,
``xv6, a simple Unix-like teaching operating system'', page 54, Nov., 2018,
\url{https://pdos.csail.mit.edu/6.828/2014/xv6/book-rev8.pdf}

\end{thebibliography}


\end{document}
