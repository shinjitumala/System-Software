\documentclass[a4j,11pt]{jarticle}
\usepackage{epsfig,here}
\usepackage{url}

\usepackage{color}
\usepackage{xcolor}
\usepackage{listings, jlisting}

\definecolor{javared}{rgb}{0.6,0,0} % for strings
\definecolor{javagreen}{rgb}{0.25,0.5,0.35} % comments
\definecolor{javapurple}{rgb}{0.5,0,0.35} % keywords
\definecolor{javadocblue}{rgb}{0.25,0.35,0.75} % javadoc
\colorlet{light-gray}{gray!20}

\lstset{
  %プログラム言語(複数の言語に対応,C,C++も可)
  language = C,
  %背景色と透過度
  backgroundcolor={\color[gray]{.90}},
  %枠外に行った時の自動改行
  breaklines = true,
  %自動開業後のインデント量(デフォルトでは20[pt])
  breakindent = 10pt,
  %標準の書体
  basicstyle = \ttfamily\scriptsize,
  %basicstyle = {\small}
  %コメントの書体
  commentstyle = {\itshape \color[cmyk]{1,0.4,1,0}},
  %関数名等の色の設定
  classoffset = 0,
  %キーワード(int, ifなど)の書体
  keywordstyle = {\bfseries \color[cmyk]{0,1,0,0}},
  %""で囲まれたなどの"文字"の書体
  % stringstyle = {\ttfamily \color[rgb]{0,0,1}},
  %枠 "t"は上に線を記載, "T"は上に二重線を記載
  %他オプション:leftline,topline,bottomline,lines,single,shadowbox
  frame = TBrl,
  %frameまでの間隔(行番号とプログラムの間)
  framesep = 5pt,
  %行番号の位置
  numbers = left,
  %行番号の間隔
  stepnumber = 1,
  %右マージン
  %xrightmargin=0zw,
  %左マージン
  %xleftmargin=3zw,
  %行番号の書体
  numberstyle = \tiny,
  %タブの大きさ
  tabsize = 4,
  %キャプションの場所("tb"ならば上下両方に記載)
  captionpos = t,
  showstringspaces=false
}

\setlength{\textwidth}{1.1\textwidth}
\setlength{\oddsidemargin}{-3pt}
\setlength{\evensidemargin}{\oddsidemargin}
\setlength{\topmargin}{10mm}
\setlength{\headheight}{0mm}
\setlength{\headsep}{0mm}

\begin{document}

\begin{center}
%\noindent
 \vspace{10mm}

{\bf {\huge システムソフトウェア 小課題1: レポート}}
%\end{center}

\vspace{80mm}

提出日:2018年10月26日

\vspace{10mm}

系/学科/類:情報工学系

\vspace{10mm}

学籍番号:16B13354

\vspace{10mm}

ログイン名:hoshino.s.af

\vspace{20mm}

{\bf {\LARGE 氏名:星野 シンジ}}
\end{center}

\newpage

\section{sleep\_sec()システムコールの動作概要}

まず、最初にcmostimeを用いて現在のrtcdate構造体を取得する。
その上で、得られたrtcdateの構造体から引数秒後
のrtcdate構造体を計算する。このrtcdate構造体は変数として持っておく。

次に、実際にプロセスをスリープロックさせる部分だが、最初に
tickslockをacquireしてスピンロックの確保している。
そして、スピンロックを確保した後は、while文が40回ループするごとにcmostimeで現在の時刻を確認し、
先に計算した引数秒後のrtcdateと等しい、またはそれよりも未来になれば、while文を脱する
ように実装した。また、ループする時に毎回sleepを呼び出している。

最後に、while文を抜けた後は、releaseでtickslockを解放している。

ソースコードの一部を図\ref{sleepsec}に示しておく。

\begin{figure}[H]
  \begin{lstlisting}
    int sys_sleep_sec(void) {
      int n;
      struct rtcdate dp0, dp;

      // returns error if the int argument was negative
      if (argint(0, &n) < 0)
        return -1;

      cmostime(&dp0);

      // calculate the rtcdate after n seconds
      dp0.second += n;
      // ... (長いので省略)

      int l = 0;
      // sleep until the current rtcdate dp is ahead of dp0
      acquire(&tickslock);
      while (1) {
        if (myproc()->killed) {
          release(&tickslock);
          return -1;
        }

        l++;
        l = l % 40;
        if (l == 0) { // only call costime every 40 ticks to save resource
          cmostime(&dp);
          if (dp.second >= dp0.second && dp.minute >= dp0.minute &&
              dp.hour >= dp0.hour && dp.day >= dp0.day && dp.month >= dp0.month &&
              dp.year >= dp0.year) {
            break;
          }
        }
        sleep(&ticks, &tickslock);
      }
      release(&tickslock);

      return 0;
    }
  \end{lstlisting}
  \caption{sleep\_sec()システムコールのコード一部}
  \label{sleepsec}
\end{figure}

\section{実装についての考察}

sleep\_secのシステムコールが必要とする資源は、
uint型のticksとspinlock構造体のtickslockである。
それらを使用している間は、tickslockをaquireすることによって、
排他制御をしている。
従って、sleep\_secのシステムコールがOSの他の動作に影響を与えることは
基本的にない。
ただし、計算資源を使うので、それに関連する影響は生じる可能性がある。


計算資源の利用ができるだけ少なくなるようにいくつか工夫した。


まず、引数秒後のrtcdateの計算をする部分では、
sleep\_sec()の引数が小さい場合に、無駄な計算をすることがないように、
秒の計算から行なっている。
具体的には、最初にrtcdateのsecondに引数を足す。その結果secondが60
を超えていれば、minuteに60で割った商を加え、secondは余りとする。
これと同じように、年まで計算することにより、秒より単位が大きい時間は
必要がある時にだけ計算されるようになっている。


次に、引数秒後の時間になるまでのwhile文については、
cmostimeの実行に計算資源が多く必要であることを考慮し、
40ticsk(約一秒)ごとにのみ、cmostimeを呼び出し及びループの抜け出しの条件確認をするようにしている。

\section{実装までの経緯}

スリープロックで引数秒間プロセスをスリープさせるsleep\_secシステムコールを
実装するというのが今回の課題だった。
よって、スリープロックがxv6のどこかに実装がないかを探すところから始め、
sleepシステムコールを見つけた。
すでに実装されているsleepシステムコールの引数をtick数ではなく、
秒数に変更するという方針で実装を試みた。
従って、sleep\_secシステムコールの基本的な構造は、sleepシステムコールと同じになり、
rtcdate構造体とcmostimeの扱いが主な注目点となった。
特に、引数秒後のrtcdate構造体の計算をする際に、うるう年等の処理のコードが最も長い。

\end{document}
