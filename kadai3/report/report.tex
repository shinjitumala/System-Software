\documentclass[a4j,11pt]{jarticle}
\usepackage{epsfig,here}
\usepackage{url}

\usepackage{color}
\usepackage{xcolor}
\usepackage{listings, jlisting}

% \definecolor{javared}{rgb}{0.6,0,0} % for strings
% \definecolor{javagreen}{rgb}{0.25,0.5,0.35} % comments
% \definecolor{javapurple}{rgb}{0.5,0,0.35} % keywords
% \definecolor{javadocblue}{rgb}{0.25,0.35,0.75} % javadoc
% \colorlet{light-gray}{gray!20}
%
\lstset{
  %プログラム言語(複数の言語に対応,C,C++も可)
  % language = Java,
  %背景色と透過度
  backgroundcolor={\color[gray]{.90}},
  %枠外に行った時の自動改行
  breaklines=true,
  %自動開業後のインデント量(デフォルトでは20[pt])
  breakindent = 10pt,
  %標準の書体
  basicstyle = \ttfamily\scriptsize,
  %basicstyle = {\small}
  %コメントの書体
  commentstyle = {\itshape \color[cmyk]{1,0.4,1,0}},
  %関数名等の色の設定
  classoffset = 0,
  %キーワード(int, ifなど)の書体
  keywordstyle = {\bfseries \color[cmyk]{0,1,0,0}},
  %""で囲まれたなどの"文字"の書体
  % stringstyle = {\ttfamily \color[rgb]{0,0,1}},
  %枠 "t"は上に線を記載, "T"は上に二重線を記載
  %他オプション:leftline,topline,bottomline,lines,single,shadowbox
  frame = shadowbox,
  %frameまでの間隔(行番号とプログラムの間)
  framesep = 5pt,
  %行番号の位置
  % numbers = left,
  %行番号の間隔
  % stepnumber = 1,
  %右マージン
  %xrightmargin=0zw,
  %左マージン
  %xleftmargin=3zw,
  %行番号の書体
  numberstyle = \tiny,
  %タブの大きさ
  tabsize = 4,
  %キャプションの場所("tb"ならば上下両方に記載)
  captionpos = t,
  showstringspaces=false
}

\setlength{\textwidth}{1.1\textwidth}
\setlength{\oddsidemargin}{-3pt}
\setlength{\evensidemargin}{\oddsidemargin}
\setlength{\topmargin}{10mm}
\setlength{\headheight}{0mm}
\setlength{\headsep}{0mm}

\begin{document}

\begin{center}
%\noindent
 \vspace{10mm}

{\bf {\huge システムソフトウェア 大課題}}
%\end{center}

\vspace{80mm}

提出日:2018年12月3日

\vspace{10mm}

系/学科/類:情報工学系

\vspace{10mm}

学籍番号:16B13354

\vspace{10mm}

ログイン名:hoshino.s.af

\vspace{20mm}

{\bf {\LARGE 氏名:星野 シンジ}}
\end{center}

\newpage

\section{テストプログラムの実装}

スーパーブロックに関する一貫性、ブロックの使用状況に関する一貫性、
inodeに関する一貫性、ディレクトリに関する一貫性をすべて実装した。
ソースコードは、全てsrcフォルダ内に入れた。

テストの結果のまとめは、標準出力に出力され、
個々のエラーは標準エラーに出力されるので、
リダイレクトを適切に用いて、見たい情報を分けられる。
今回のテストの実行結果は、標準出力のものをtest08.txtとして、
標準出力と標準エラーをまとめたものは、test08\_extended.txtとして
zipファイルに含めている。

実行するときの引数で、動作が異なる。
引数なしで実行すれば、
sampleフォルダ内にあるfs01.imgからfs12.imgまで
のイメージファイルを自動的にテストし、
標準出力と標準エラーに結果を出力する。

"raw"を引数として与えれば、
sampleフォルダ内にあるfs01.imgからfs12.imgまで
のイメージファイルの中身を4バイトずつ読み込み、
unsigned intとして、1ブロック1行でファイル出力する。
ファイルは、output\\fs01\_raw.txtという形式で
各イメージファイルについて出力される。

イメージファイルへのパスを引数として与えれば、
そのイメージファイル単独のテストを行い、
結果を標準出力と標準エラー
に出力する。

\section{実行結果}

テスト結果のまとめは、標準出力に出力し、より細かいエラー原因は
標準エラーに出力した。
以下に、標準出力に出された結果を示す。

\lstinputlisting{test07.txt}

ここで、細かいエラーメッセージで分かったことに基づいて、
各イメージファイルについて、どこが壊れているか以下で説明する。

\subsection*{fs01.img}

スーパーブロックのinodestartが間違っているために、連鎖的にファイルの読み込みで
エラーが起きている。

\subsection*{fs02.img}

bmapに一か所誤りが存在する。bmapでは、600番目のブロックが使用済み
ということになっているが、どのinodeにも参照されていない。

\subsection*{fs03.img}

エラーはなく、一貫性を保てている。

\subsection*{fs04.img}

一つのinodeに誤りがあり、bmapで未使用になっているブロックを一か所差している。
誤りのあるinodeは、それ以降のテストで除外されてしまうので、
それ以外のエラーは、そのinodeが除外されたことから生じている。

\subsection*{fs05.img}

スーパーブロックのsizeの値が破壊されている。
データブロックの範囲を$size - nblocks$で計算しているために、
連鎖的にほかの部分でもエラーが生じている。

\subsection*{fs06.img}

スーパーブロックのninodesの値が間違っている。
そのため、テストプログラムでは、inodeに必要なブロック数から計算されたbmapstartと
実際のbmapstartにずれが生じ、bmapstartでエラーがあるという報告になった。
更に、bmapブロックの内容をinodeとして解釈するので、inodeのエラーも大量に発生した。

\subsection*{fs07.img}

1つのinodeのsizeが間違っている。
副作用は、fs04.imgと同じ理由で生じており、
inodeのエラーが検知されるとそれ以降のテストでは、
そのinodeは除外されるから、ほかのエラーも現れている。

\subsection*{fs08.img}

検知されているエラーは、1つ目がブロック61がbmapで使用済みになっているのに、
どのinodeにも参照されていない、2つ目がブロック63が複数のinodeから
参照されているということであった。

このことから、1つのinodeの参照するブロックがは破壊され、
61から63に変更されたということが考えられる。

\subsection*{fs09.img}

1つのinodeのnlinkの値が間違っている。

\subsection*{fs10.img}

1つのinodeの参照しているブロックに
bmapで未使用になっているブロック800
を参照しているものがある。

それ以外のエラーは、そのエラーにより
inodeが除外されたこのによる副作用である。

\subsection*{fs11.img}

ルートディレクトリの".."が自分自身を指しておらず、
18番目のinodeを指している。さらに、18番目のinodeの
".."がルートディレクトリを指している。
これは、ルートディレクトリの一貫性が保たれていないだけでなく、
お互いがお互いの親という関係になっており、
矛盾している。

\subsection*{fs12.img}

スーパーブロックのsizeの値が間違っている。

inodeの参照しているブロックがデータブロックを指しているかを
チェックするときに$size - nblocks$でデータブロックの始まり
を計算しているので、間違ったsizeの値により、
テストが誤って複数のinodeを破壊されていると判断してしまっている。
ほかのエラーはそれによるものである。

\end{document}
